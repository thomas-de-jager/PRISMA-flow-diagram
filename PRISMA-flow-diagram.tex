\documentclass[tikz,border=10pt]{standalone}
\usetikzlibrary{positioning,chains,scopes}

\begin{document}
    \tikzset{
        mynode/.style={
            draw,
            rectangle,
            align=center,
            text width=25mm,
            minimum height=15mm,
            font=\footnotesize,
            inner sep=1mm,
        },
        mynoderight/.style={
            draw,
            rectangle,
            align=center,
            text width=15mm,
            minimum height=15mm,
            font=\footnotesize,
            inner sep=1mm,
        },
        mylabel/.style={
            draw,
            rectangle,
            align=center,
            rounded corners,
            font=\footnotesize\bf,
            inner sep=1mm,
            fill=cyan!30,
            minimum height=20mm
        },
        arrow/.style={
            very thick,
            ->,
            >=stealth
        }
    }

    \begin{tikzpicture}[
        node distance=6mm,
        start chain=1 going below,
        every join/.style=arrow,
        ]

        
        % Chain in the center going below
        \coordinate[on chain=1] (tc);
        \node[mynode, on chain=1] (n2)
            {Records after duplicates removed\\(n=3655)}; % calculate from uniqueness percentage in table and add result of 2nd table 
        \node[mynode, join, on chain=1] (n3)
            {Records screened\\(n=3671)}; % same as above (and additional records node)
        \node[mynode, join, on chain=1] (n4)
            {Full-text articles analysed\\(n=342)}; % mendeley folders: msc lit. rev. + interesting/not relevant + deleted documents
        \node[mynode, join, on chain=1] (n5)
            {Studies included in literature review\\(n=163)}; % total amount in msc. literature review folder mendeley
        % \node[mynode, join, on chain=1] (n5)
            % {Studies included in qualitative synthesis\\(n=...)};
        % \node[mynode, join, on chain=1] (n6)
            % {Studies included in quantitative synthesis (meta-analysis)\\(n=...)};
    
        % Branches to the right
        \begin{scope}[start chain=going right]
            \chainin (n3);
            \node[mynoderight, join, on chain]
                {Records excluded\\(n=3329)}; % calculate: records screened - full text articles analysed
            \chainin (n4);
            \node[mynoderight, join, on chain]
                {Full-text articles excluded\\(n=179)}; % calculate: full text articles analysed - studies included in lit. rev.
        \end{scope}
    
        % Nodes at the top  
        \node[mynode, left=4mm of tc, anchor=south east] (n1l)
            {Records identified through database searching\\(n=4040)}; % from 2 tables total amount
        \node[mynode, right=4mm of tc, anchor=south west] (n1r)
            {Additional records identified through other sources\\ (n=16)}; % approximated
    
        \coordinate (n2nl) at ([xshift=-10mm]n2.north);
        % \coordinate (n2nr) at ([xshift= 15mm]n2.north);
        \coordinate (n1rr) at ([xshift= 0mm]n1r.south);
        \coordinate (n2n3) at ([yshift= -2.5mm]n2.south);
        \coordinate (n1rs) at ([yshift= -23.65mm, xshift=0mm]n1r.south);
        \draw[arrow] (n1l.south -| n2nl) -- (n2nl);
        % \draw[arrow] (n1r.south -| n2nr) -- (n2nr);
        \draw[arrow] (n1r.south -| n1rr)-- (n1rs) -- (n2n3);
    
        % Labels on the left
        \begin{scope}[start chain=going below, xshift=-40mm, node distance=8mm]
            \node[mylabel, minimum width=9mm, minimum height=19mm, yshift=8mm, on chain, inner sep=2pt] {\rotatebox{90}{Identification}};
            \node[mylabel, minimum width=9mm, minimum height=38mm, yshift=5mm, on chain] {\rotatebox{90}{Screening}};
            \node[mylabel, minimum width=9mm, minimum height=19mm, yshift=5mm, on chain, inner sep=2pt] {\rotatebox{90}{Eligibility}};
            \node[mylabel, minimum width=9mm, minimum height=19mm, yshift=5mm, on chain] {\rotatebox{90}{Included}};
        \end{scope}
    
        % Title
        \node[above=20mm of tc, font=\normalsize\bf] {PRISMA Flow Diagram};
    \end{tikzpicture}
\end{document}
